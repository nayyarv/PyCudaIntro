%% LyX 2.1.3 created this file.  For more info, see http://www.lyx.org/.
%% Do not edit unless you really know what you are doing.

\documentclass[english]{beamer}
\usepackage[T1]{fontenc}
\usepackage[latin9]{inputenc}
\usepackage{amsmath}

\makeatletter
%%%%%%%%%%%%%%%%%%%%%%%%%%%%%% Textclass specific LaTeX commands.
 % this default might be overridden by plain title style
 \newcommand\makebeamertitle{\frame{\maketitle}}%
 % (ERT) argument for the TOC
 \AtBeginDocument{%
   \let\origtableofcontents=\tableofcontents
   \def\tableofcontents{\@ifnextchar[{\origtableofcontents}{\gobbletableofcontents}}
   \def\gobbletableofcontents#1{\origtableofcontents}
 }

%%%%%%%%%%%%%%%%%%%%%%%%%%%%%% User specified LaTeX commands.
\usetheme{Rochester}
\usecolortheme{beaver}

\usenavigationsymbolstemplate{}
\setbeamercolor{structure}{fg=darkred}
\setbeamercolor{block body}{bg=gray!10!white}

\makeatother

\usepackage{babel}
\begin{document}
\title{Intro to PyCuda}
\subtitle{For Bayesian Stats and Gaussian Mixture Models}


\author{Varun Nayyar}


\date{27/07/18}
\makebeamertitle


\begin{frame}[t]\frametitle{Motivating Problem}

\begin{block}{Motivation}
    ABCs have found widespread use in many areas of computational biology and population genetics.

    In the case of contamination models, 
    \begin{align*}
        p(x) &= (1-\pi)p(x|\theta_{main}) + \pi p(x|\theta_c)
    \end{align*}
    These are easily modelled using mixture models, but there are limited methods available for analysing contamination models with intractable likelihoods.
\end{block}
    
\end{frame}














\end{document}

