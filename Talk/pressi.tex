%% LyX 2.1.3 created this file.  For more info, see http://www.lyx.org/.
%% Do not edit unless you really know what you are doing.

\documentclass[english]{beamer}
\usepackage[T1]{fontenc}
\usepackage[latin9]{inputenc}
\usepackage{amsmath}

\makeatletter
%%%%%%%%%%%%%%%%%%%%%%%%%%%%%% Textclass specific LaTeX commands.
 % this default might be overridden by plain title style
 \newcommand\makebeamertitle{\frame{\maketitle}}%
 % (ERT) argument for the TOC
 \AtBeginDocument{%
   \let\origtableofcontents=\tableofcontents
   \def\tableofcontents{\@ifnextchar[{\origtableofcontents}{\gobbletableofcontents}}
   \def\gobbletableofcontents#1{\origtableofcontents}
 }

%%%%%%%%%%%%%%%%%%%%%%%%%%%%%% User specified LaTeX commands.
\usetheme{Rochester}
\usecolortheme{beaver}

\usenavigationsymbolstemplate{}
\setbeamercolor{structure}{fg=darkred}
\setbeamercolor{block body}{bg=gray!10!white}

\makeatother

\usepackage{babel}
\begin{document}
\title{Accelerating your Python Code}
\subtitle{For GMMs with PyCUDA}

\author{Varun Nayyar}


\date{27/07/18}
\makebeamertitle

\begin{frame}[t]\frametitle{Outline}
Me = Math Major + Script Kiddy (Manage Expectations)
 
\begin{block}{What to Expect}
  \begin{itemize}
    \item Some Math
    \item Iterative Process I went through
    \item Thinking with CUDA and Basic Syntax
    \item How to use PyCUDA to avoid complicated work
  \end{itemize}

\end{block}

\begin{block}{What NOT to Expect}
  \begin{itemize}
    \item How PyCUDA does it's magic
    \item Intermediate/Advanced CUDA
  \end{itemize}
\end{block}


\end{frame}



\begin{frame}[t]\frametitle{Gaussian Mixture Models d=1, K=2}
    \begin{center}
        \includegraphics[width=10cm]{img/Combined.png}
    \end{center}
K-means+=1
\end{frame}


\begin{frame}[t]\frametitle{Gaussian Mixture Models (GMMs)}
\begin{block}{Density Function}
For $K$ mixtures
\begin{align*}
f(\mathbf{x}|\boldsymbol{\pi},\boldsymbol{\mu},\boldsymbol{\Sigma})= & \sum_{k=1}^{K}\pi_{k}\mathcal{N}(\mathbf{x}|\mu_{k},\Sigma_{k}) 
\end{align*}
\end{block}

\begin{block}{Log Likelihood (function of concern)}
  \begin{align*}
  l(\mu,\Sigma,\mathbf{x})  = &\sum_{i=1}^{N}\ln\left(\sum_{k=1}^{K}\pi_{k}\mathcal{N}(x_{i}|\mu_{k},\Sigma_{k})\right)\label{eq:LogLikelihood}
  \end{align*}
\end{block}

\end{frame}


\begin{frame}[t]\frametitle{Need for speed}
Some post-hoc realizations
\begin{itemize}
    \item GMM likelihood formula doesn't decompose into a mathematically simple form
    \item However, note that the GMM likelihood has a parallelizable form in that each point of each mixture is independent (CUDA vibes)
\end{itemize}
Computational Numbers
\begin{itemize}
    \item Number of flops are of the order of $O(NKd)$, in my case, $N = 10^6$, $K = 8$, $d = 13$. I.e. $O(10^8)$
    \item I needed to evaluate the likelihood $10^6$ times for a fixed dataset while the parameters were varied. (Markov Chain Monte Carlo)
    \item i.e $10^{14}$ floating point operations per run.
\end{itemize}        
\end{frame}

\begin{frame}[t]\frametitle{First Attempt}

\begin{columns}

  \column{0.5\textwidth}
    \begin{itemize}
      \item Eh, my computer is fast enough
      \item Pure Python (numpy)
      \item Took 36 us per datapoint, or 36s for whole dataset
      \item $10^6$ evaluations would take ~1 year
    \end{itemize}


  \column{0.5\textwidth}
    \begin{center}
        \includegraphics[height=6cm]{img/doingitlive.jpg}
    \end{center}

\end{columns}

\end{frame}


\begin{frame}[t]\frametitle{Execution speed per N}
    \begin{center}
        \includegraphics[width=10cm]{img/simpleOnly.png}
    \end{center}
\end{frame}


\begin{frame}[t]\frametitle{Second Attempt}

\begin{columns}

  \column{0.5\textwidth}

    \begin{itemize}
      \item Stand on the shoulders of giants (scikit-learn)
      \item Reverse engineered the likelihood evaluator
      \item 72x improvement!
      \item Took 0.5 us per datapoint, or 5s for whole dataset
      \item $10^6$ evaluations would take ~5 days!
    \end{itemize}


  \column{0.5\textwidth}
    \begin{center}
        \includegraphics[width=4cm,height=6cm]{img/shoulders.jpg}
    \end{center}
\end{columns}

\end{frame}


\begin{frame}[t]\frametitle{Execution speed per N}
    \begin{center}
        \includegraphics[width=10cm]{img/skandsimple.png}
    \end{center}
\end{frame}



\end{document}

